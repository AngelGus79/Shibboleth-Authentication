\label{sec:introduction}
\section{Introduction}
User authentication and user authorization are main tasks needed to guarantee security in IT environments.
One important requirement of such security solutions, is that they must be as transparent as possibile.
In systems where resources sit on different systems and environments, which are a common reality for the users, a solution must be found to eliminate the need for each system
to create and manage its own credential system.
For such reason single sign-on (SSO, as described in \cite{Shaer-1995}) systems have been designed and implemented to achieve such goal.\\
\\
One of the most widespread SSO software package is Shibboleth \cite{Morgan-2004}, which implements authentication and authorization over network resources with a federated
approach.
Shibboleth, designed and implemented by Internet2, provides an effective solution for secure multi-organizational access to web resources.
It implements widely used federated identity standards, principally OASIS' Security Assertion Markup Language (SAML \cite{Cantor-2005}), to provide a federated single
sign-on and attribute exchange framework.
Shibboleth also provides extended privacy functionality allowing the browser user and their home site to control the attributes released to each application.\\
\\
Using Shibboleth-enabled access simplifies management of identity and permissions for organizations supporting users and applications.
Shibboleth is developed in an open and participatory environment and is freely available.
For these reason it has formed the heart of many solutions for single sign-on and Authentication \& Authorization Infrastructures (AAI) in real identity federations.\\
\\
The architecture of Shibboleth, described in \cite{Erdos-2005}, identifies two main components: Identity Providers (IdPs) which authenticate the users and supply user 
authorization attributes; the Service Providers (SPs) which consume user attributes and provide access to the secure contents.\\
The communication excanges between the user, the IdP and the SP are all base don SAML assertions and transit over HTTP channels.
This implementation architecture permits Shibboleth to works seamlessly inside user web-browsers.
However, this approach is strongly web-centric, and for this reason does not adapt well to provide SSO authentication to non web-based application.\\
\\
In this article a solution is presented to permit Shibboleth authentication for non web-based application.
Different implementations will be described showing how to integrate the shibboleth AAI mechanisms inside several client applications (Linux platform, Java JAAS, Python).
The rest of this paper is organized as follows: the first section contains a quick overview of related works; the second section will present the architecture of
the solution presented with this article; the third section will present some details about the technical implementation in the different software platform; the last
section presents conclusions and future works.

\label{sec:relatedworks}
\section{Related works}

\label{sec:architecture}
\section{Architecture}

\label{sec:implementations}
\section{Implementations}

\label{sec:pamnss}
\subsection{PAM and NSS modules}

\label{sec:jaas}
\subsection{JAAS module}

\label{sec:python}
\subsection{Python}

\label{sec:conclusion}
\section{Conclusion and future works}

